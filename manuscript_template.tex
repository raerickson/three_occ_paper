% Options for packages loaded elsewhere
\PassOptionsToPackage{unicode}{hyperref}
\PassOptionsToPackage{hyphens}{url}
%
\documentclass[
]{article}
\title{Combining biology and statistical hierarchicies with occpuancy
models to account for imperfect detection of single and multiple species
with two- and three-level occpuancy models}
\author{Richard A. Erickson, 0000-0003-4649-482X,
\href{mailto:rerickson@usgs.gov}{\nolinkurl{rerickson@usgs.gov}} \and Barb
A. Bennie \and Laura Pederman 0000-0001-6976-4138 \and Kevin's student
for her data? \and Steven Spear \and Kevin Lafferty 0000-0001-7583-4593}
\date{18 May 2022}

\usepackage{amsmath,amssymb}
\usepackage{lmodern}
\usepackage{iftex}
\ifPDFTeX
  \usepackage[T1]{fontenc}
  \usepackage[utf8]{inputenc}
  \usepackage{textcomp} % provide euro and other symbols
\else % if luatex or xetex
  \usepackage{unicode-math}
  \defaultfontfeatures{Scale=MatchLowercase}
  \defaultfontfeatures[\rmfamily]{Ligatures=TeX,Scale=1}
\fi
% Use upquote if available, for straight quotes in verbatim environments
\IfFileExists{upquote.sty}{\usepackage{upquote}}{}
\IfFileExists{microtype.sty}{% use microtype if available
  \usepackage[]{microtype}
  \UseMicrotypeSet[protrusion]{basicmath} % disable protrusion for tt fonts
}{}
\makeatletter
\@ifundefined{KOMAClassName}{% if non-KOMA class
  \IfFileExists{parskip.sty}{%
    \usepackage{parskip}
  }{% else
    \setlength{\parindent}{0pt}
    \setlength{\parskip}{6pt plus 2pt minus 1pt}}
}{% if KOMA class
  \KOMAoptions{parskip=half}}
\makeatother
\usepackage{xcolor}
\IfFileExists{xurl.sty}{\usepackage{xurl}}{} % add URL line breaks if available
\IfFileExists{bookmark.sty}{\usepackage{bookmark}}{\usepackage{hyperref}}
\hypersetup{
  pdftitle={Combining biology and statistical hierarchicies with occpuancy models to account for imperfect detection of single and multiple species with two- and three-level occpuancy models},
  pdfauthor={Richard A. Erickson, 0000-0003-4649-482X, rerickson@usgs.gov; Barb A. Bennie; Laura Pederman 0000-0001-6976-4138; Kevin's student for her data?; Steven Spear; Kevin Lafferty 0000-0001-7583-4593},
  hidelinks,
  pdfcreator={LaTeX via pandoc}}
\urlstyle{same} % disable monospaced font for URLs
\usepackage[margin=1in]{geometry}
\usepackage{color}
\usepackage{fancyvrb}
\newcommand{\VerbBar}{|}
\newcommand{\VERB}{\Verb[commandchars=\\\{\}]}
\DefineVerbatimEnvironment{Highlighting}{Verbatim}{commandchars=\\\{\}}
% Add ',fontsize=\small' for more characters per line
\usepackage{framed}
\definecolor{shadecolor}{RGB}{248,248,248}
\newenvironment{Shaded}{\begin{snugshade}}{\end{snugshade}}
\newcommand{\AlertTok}[1]{\textcolor[rgb]{0.94,0.16,0.16}{#1}}
\newcommand{\AnnotationTok}[1]{\textcolor[rgb]{0.56,0.35,0.01}{\textbf{\textit{#1}}}}
\newcommand{\AttributeTok}[1]{\textcolor[rgb]{0.77,0.63,0.00}{#1}}
\newcommand{\BaseNTok}[1]{\textcolor[rgb]{0.00,0.00,0.81}{#1}}
\newcommand{\BuiltInTok}[1]{#1}
\newcommand{\CharTok}[1]{\textcolor[rgb]{0.31,0.60,0.02}{#1}}
\newcommand{\CommentTok}[1]{\textcolor[rgb]{0.56,0.35,0.01}{\textit{#1}}}
\newcommand{\CommentVarTok}[1]{\textcolor[rgb]{0.56,0.35,0.01}{\textbf{\textit{#1}}}}
\newcommand{\ConstantTok}[1]{\textcolor[rgb]{0.00,0.00,0.00}{#1}}
\newcommand{\ControlFlowTok}[1]{\textcolor[rgb]{0.13,0.29,0.53}{\textbf{#1}}}
\newcommand{\DataTypeTok}[1]{\textcolor[rgb]{0.13,0.29,0.53}{#1}}
\newcommand{\DecValTok}[1]{\textcolor[rgb]{0.00,0.00,0.81}{#1}}
\newcommand{\DocumentationTok}[1]{\textcolor[rgb]{0.56,0.35,0.01}{\textbf{\textit{#1}}}}
\newcommand{\ErrorTok}[1]{\textcolor[rgb]{0.64,0.00,0.00}{\textbf{#1}}}
\newcommand{\ExtensionTok}[1]{#1}
\newcommand{\FloatTok}[1]{\textcolor[rgb]{0.00,0.00,0.81}{#1}}
\newcommand{\FunctionTok}[1]{\textcolor[rgb]{0.00,0.00,0.00}{#1}}
\newcommand{\ImportTok}[1]{#1}
\newcommand{\InformationTok}[1]{\textcolor[rgb]{0.56,0.35,0.01}{\textbf{\textit{#1}}}}
\newcommand{\KeywordTok}[1]{\textcolor[rgb]{0.13,0.29,0.53}{\textbf{#1}}}
\newcommand{\NormalTok}[1]{#1}
\newcommand{\OperatorTok}[1]{\textcolor[rgb]{0.81,0.36,0.00}{\textbf{#1}}}
\newcommand{\OtherTok}[1]{\textcolor[rgb]{0.56,0.35,0.01}{#1}}
\newcommand{\PreprocessorTok}[1]{\textcolor[rgb]{0.56,0.35,0.01}{\textit{#1}}}
\newcommand{\RegionMarkerTok}[1]{#1}
\newcommand{\SpecialCharTok}[1]{\textcolor[rgb]{0.00,0.00,0.00}{#1}}
\newcommand{\SpecialStringTok}[1]{\textcolor[rgb]{0.31,0.60,0.02}{#1}}
\newcommand{\StringTok}[1]{\textcolor[rgb]{0.31,0.60,0.02}{#1}}
\newcommand{\VariableTok}[1]{\textcolor[rgb]{0.00,0.00,0.00}{#1}}
\newcommand{\VerbatimStringTok}[1]{\textcolor[rgb]{0.31,0.60,0.02}{#1}}
\newcommand{\WarningTok}[1]{\textcolor[rgb]{0.56,0.35,0.01}{\textbf{\textit{#1}}}}
\usepackage{graphicx}
\makeatletter
\def\maxwidth{\ifdim\Gin@nat@width>\linewidth\linewidth\else\Gin@nat@width\fi}
\def\maxheight{\ifdim\Gin@nat@height>\textheight\textheight\else\Gin@nat@height\fi}
\makeatother
% Scale images if necessary, so that they will not overflow the page
% margins by default, and it is still possible to overwrite the defaults
% using explicit options in \includegraphics[width, height, ...]{}
\setkeys{Gin}{width=\maxwidth,height=\maxheight,keepaspectratio}
% Set default figure placement to htbp
\makeatletter
\def\fps@figure{htbp}
\makeatother
\setlength{\emergencystretch}{3em} % prevent overfull lines
\providecommand{\tightlist}{%
  \setlength{\itemsep}{0pt}\setlength{\parskip}{0pt}}
\setcounter{secnumdepth}{5}
\newlength{\cslhangindent}
\setlength{\cslhangindent}{1.5em}
\newlength{\csllabelwidth}
\setlength{\csllabelwidth}{3em}
\newlength{\cslentryspacingunit} % times entry-spacing
\setlength{\cslentryspacingunit}{\parskip}
\newenvironment{CSLReferences}[2] % #1 hanging-ident, #2 entry spacing
 {% don't indent paragraphs
  \setlength{\parindent}{0pt}
  % turn on hanging indent if param 1 is 1
  \ifodd #1
  \let\oldpar\par
  \def\par{\hangindent=\cslhangindent\oldpar}
  \fi
  % set entry spacing
  \setlength{\parskip}{#2\cslentryspacingunit}
 }%
 {}
\usepackage{calc}
\newcommand{\CSLBlock}[1]{#1\hfill\break}
\newcommand{\CSLLeftMargin}[1]{\parbox[t]{\csllabelwidth}{#1}}
\newcommand{\CSLRightInline}[1]{\parbox[t]{\linewidth - \csllabelwidth}{#1}\break}
\newcommand{\CSLIndent}[1]{\hspace{\cslhangindent}#1}
\usepackage{setspace}\doublespacing \usepackage{lineno} \linenumbers
\usepackage{subfig}
\ifLuaTeX
  \usepackage{selnolig}  % disable illegal ligatures
\fi

\begin{document}
\maketitle

\textbf{This draft manuscript is distributed solely for purposes of
scientific peer review. Its content is deliberative and predecisional,
so it must not be disclosed or released by reviewers. Because the
manuscript has not yet been approved for publication by the U.S.
Geological Survey (USGS), it does not represent any official USGS
finding or policy.}

\hypertarget{abstract}{%
\section{Abstract}\label{abstract}}

\hypertarget{introduction}{%
\section{Introduction}\label{introduction}}

\begin{itemize}
\item
  Problem of imperfect detection

  \begin{itemize}
  \tightlist
  \item
    May miss seeing a species due to low detection
  \item
    Occupancy models allow for imperfect detection to modeled and
    account for zero-inflation (MacKenzie \emph{et al.} 2017)
  \item
    Large application in ecological theory and practices, with text
    books written on the topic (e.g., Royle \& Dorazio 2008; MacKenzie
    \emph{et al.} 2017)
  \end{itemize}
\item
  Multi-species models allow for the co-occurrence of species to be
  modeled.

  \begin{itemize}
  \item
    Species often best predictors of other species
  \item
    Lack environmental predictor variables
  \item
    Tobler \emph{et al.} (2019) describe deriving two methods for
    deriving multi-species models

    \begin{itemize}
    \tightlist
    \item
      Join species distribution models based upon an extension of the
      regression framework used for single species models, starting with
      Dorazio \& Royle (2005a).
    \item
      Multi-species models that stack together single species models by
      explicit capturing pair-wise correlations starting with Latimer
      \emph{et al.} (2009)
    \item
      Tobler \emph{et al.} (2019) present a framework for estimating
      species co-occurrence (or correlation of occupancy) through the
      use of latent variables.
    \end{itemize}
  \end{itemize}
\item
  we expand upon the notation of Royle \& Dorazio (2008) to incorporate
  occupancy models into a statistical hierarchical modeling framework.

  \begin{itemize}
  \tightlist
  \item
    Bridges statistical hierarchical models such as those covered in
    Gelman \& Hill (2006) with the biological hierarchy captured
    described by Royle \& Dorazio (2008).
  \item
    Directly building upon Tobler \emph{et al.} (2019) because we
    directly estimate correlation without latent variables
  \item
    Expand to include not only correlated site-level occupancy, but also
    detection probabilities
  \item
    Show how model may be extended to three-level models that account
    for sub-sampling, following Dorazio \& Erickson (2018) and their
    style application to eDNA.
  \end{itemize}
\item
  Motivating

  \begin{itemize}
  \tightlist
  \item
    2-level, motivation for Correlations among detection of parasites
    when detection probs correlated across hosts
  \item
    3-level motivated by eDNA monitoring program, where we have
    site-revisits and care about sampling probabilities changing through
    time for single species
  \item
    3-level community analysis for metabarcoding
  \item
    require scaleable computing, recently possibly with advances in the
    Stan language that allow for within chain parallel computing for
    HMC.
  \end{itemize}
\item
  purpose and overview of this paper

  \begin{itemize}
  \tightlist
  \item
    present models
  \item
    numeric implementation
  \item
    application to example studies
  \item
    future works
  \end{itemize}
\end{itemize}

\hypertarget{models}{%
\section{Models}\label{models}}

\hypertarget{basic-2-level-model}{%
\subsection{Basic 2-level model}\label{basic-2-level-model}}

\begin{itemize}
\item
  Dorazio--Royle multispecies occupancy model (Dorazio \& Royle 2005b),
\item
  using similar notation as (Dorazio \& Erickson 2018) because we
  generalize the model to include eDNA.
\item
  Hierarchical model based upon Gelman \& Hill (2006), described in Stan
  Development Team (2022) \S 1.13 and similar to the implementation in
  the \texttt{fishStan} package (Erickson \emph{et al.} 2020; Erickson
  \emph{et al.} 2022)
\item
  Start by presenting basic models, without formal indexing until we get
  the final model.
\item
  Observed unit is occupied (\(Z = 1\)) or not occupied (\(Z = 0\))
\item
  Probability of a unit occupied is \(\psi\)
\item
  \textbf{Is species at site?}
\end{itemize}

\begin{align}
Z &\sim \textrm{Bernoulli}(\psi)
\end{align}

\begin{itemize}
\tightlist
\item
  \textbf{Was species detected during visit?}
\item
  \(Y\) is detection (1) or non-detection (0)
\item
  the lower case \(z\) is the realized value for \(Z\)
\item
  \(k\) is number of repeated samples per site
\item
  \(p\) is probability of detection conditional that the species was
  located at the site (i.e., \(z = 1\)).
\end{itemize}

\begin{align}
Y | z &\sim \textrm{Binomail}(k, z \times p)
\end{align}

\begin{itemize}
\tightlist
\item
  This model may be expanded to include regression coefficients.
\item
  We use logit scale, other common choice is probit scale (e.g., Dorazio
  \& Erickson 2018).
\item
  Both similar, logit has slightly wider tails to the distribution
  (Finney 1952), and works more efficiently with Stan due to build in
  and optimized functions.
\end{itemize}

\begin{align}
\textrm{logit}(\psi) &= \mu_{\psi} \\
\textrm{logit}(p)  &=  \mu_{p}
\end{align}

\begin{itemize}
\tightlist
\item
  Can regressors with \(\mu\)s
\item
  Predictor matrices \(X\) and \(V\) as well a regression coefficients
  \(\beta\) and \(\delta\), that can include an error term:
\end{itemize}

\begin{align}
\mu_{\psi} & \sim X \beta \\
\mu_{p} & \sim V \delta
\end{align}

Using \(\mu_{\psi} \sim X \beta\) as an example, imagine that we fit an
intercept for each specie's detection probability at a site. For
example, the vector \(\beta\) looks like:
\(\beta_\textrm{spp 1}, \beta_\textrm{spp 2}, \ldots \beta_\textrm{spp N}\).
We can then fit a regression to each site \(i\), for example, let's say
we have \(N\) sites:

\begin{align}
\mu_{\psi_{1}} & \sim X_{1} \beta_{1} \\
\mu_{\psi_{2}} & \sim X_{2} \beta_{2} \\
\mu_{\psi_{3}} & \sim X_{3} \beta_{3} \\
\vdots
\mu_{\psi_{N}} & \sim X_{N} \beta_{N}
\end{align}

\begin{itemize}
\tightlist
\item
  We can then model the regression coefficients
\item
  We can assume both \(\beta\) and \(\delta\) come from a multivariate
  normal distribution:
\item
  Hierarchical model, adapting notation of Stan Development Team (2022)
  to use use a star symbol (\(^\star\)) for the second level hierarchy
  avoid confusion with too many parameters (i.e., the biology hierarchy
  crossed with the statistical hierarchy).
\item
  \(\Sigma\) are correlated error terms
\item
  Because each term coefficient is specices intercept, the correlation
  between these is the species' correlation across sites.
\end{itemize}

\begin{align}
\beta  & \sim \textrm{multivariate normal}(X^\star \beta^\star, \Sigma_{\psi}) \\
\delta & \sim \textrm{multivariate normal}(V^\star \delta^\star, \Sigma_{p})
\end{align}

\begin{itemize}
\tightlist
\item
  We present two models for both computational efficiency and
  statistcial identifiability, specifcially one model with
  \(\Sigma_{p}\) and one model without \(\Sigma_{p}\).
\item
  The and correlation matrices \(\Omega_\psi\) and \(\Omega_p\) are
  proportional to the covariance matrices, \(\Sigma_\psi\) and
  \(\Sigma_p\):
\end{itemize}

\begin{align}
\Omega_{\psi_{i,j}} &\propto \Sigma_{\psi_{i,j}}~\textrm{and} \\
\Omega_{p_{i,j}} &\propto \Sigma_{p_{i,j}}.
\end{align}

\hypertarget{formal-2-level-definition-and-notations}{%
\subsection{Formal 2-level definition and
notations}\label{formal-2-level-definition-and-notations}}

We base our formal nation upon Dorazio \& Erickson (2018) for the
occupancy model and Stan Development Team (2022) \S 1.13 for the
hierarchical model. We also include our Stan variable names in
\texttt{code\ format} because tracking indexing with Stan was a large
challenge we faced when implementing this model. The model has units
\(i\) that may a region of interest in spatial, temporal, or both where
\(i \in 1, 2, \ldots N_\textrm{units}\) (in Stan code, this this
\texttt{n\_units}). For example, multiple lakes could be visited, the
same lake could be visited multiple times, or multiple lakes could be
visited multiple times. The \(i^\textrm{th}\) unit may be occupied
(\(Z_i = 1\)) or not occupied (\(Z_i = 0\)) with probability \(\psi_i\):
\begin{align}
Z &\sim \textrm{Bernoulli}(\psi).
\end{align}

The \(i^\textrm{th}\) unit has \(k\) samples to the site,
\(k \in 1, 2, \ldots N_\textrm{revisits}\). \(k\) may be summed for each
unit and then written as a vector, \(K\) for all units
(\texttt{k\_samples} that is the same length as the total number of
observations in the data frame,
\(N_\textrm{total} or `total_observations`). A sampling event may have a detection (\)Y\_\{i,k\}
= 1\() or a non-detection (\)\(Y_{i,k} = 0\)). A lowercase \(z_i\)
denotes the realized occupancy (\(z_i\) = 1) or non-occupancy (\(z_i\) =
0) at a unit. The observation \(Y_{i,k}\) is conditional upon \(z_i\)
(denoted with the vertical bar symobol, \(|\)). Lastly, \(p_{i,j}\) is
the detection probability for the \(k^\textrm{th}\) sample at the
\(i^\textrm{th}\) unit.

\begin{align}
Y | z &\sim \textrm{Binomail}(k, z p)
\end{align}

\begin{itemize}
\tightlist
\item
  This model may be expanded to include regression coefficients.
\item
  We use logit scale, other common choice is probit scale (e.g., Dorazio
  \& Erickson 2018).
\item
  Both similar, logit has slightly wider tails to the distribution
  (Finney 1952), and works more efficiently with Stan due to build in
  and optimized functions.
\end{itemize}

\begin{align}
\textrm{logit}(\psi) &= \mu_{\psi} \\
\textrm{logit}(p)  &=  \mu_{p}
\end{align}

\begin{itemize}
\tightlist
\item
  Can regressors with \(\mu\)s
\item
  Predictor matrices \(X\) and \(V\) as well a regression coefficients
  \(\beta\) and \(\delta\), that can include an error term:
\end{itemize}

\begin{align}
\mu_{\psi} & \sim X \beta \\
\mu_{p} & \sim V \delta
\end{align}

\begin{itemize}
\tightlist
\item
  We can assume both \(\beta\) and \(\delta\) come from a multivariate
  normal distribution:
\item
  Hierarchical model, adapting notation of Stan Development Team (2022)
  to use use a star symbol (\(^\star\)) for the second level hierarchy
  avoid confusion with too many parameters (i.e., the biology hierarchy
  crossed with the statistical hierarchy).
\item
  \(\Sigma\) are correlated error terms
\end{itemize}

\begin{align}
\beta  & \sim \textrm{multivariate normal}(X^\star \beta^\star, \Sigma_{\psi}) \\
\delta & \sim \textrm{multivariate normal}(V^\star \delta^\star, \Sigma_{p})
\end{align}

\begin{itemize}
\tightlist
\item
  We present two models for both computational efficiency and
  statistical identifiability, specifically one model with
  \(\Sigma_{p}\) and one model without \(\Sigma_{p}\).
\item
  The and correlation matrices \(\Omega_\psi\) and \(\Omega_p\) are
  proportional to the covariance matrices, \(\Sigma_\psi\) and
  \(\Sigma_p\):
\end{itemize}

\begin{align}
\Omega_{\psi_{i,j}} &\propto \Sigma_{\psi_{i,j}}~\textrm{and} \\
\Omega_{p_{i,j}} &\propto \Sigma_{p_{i,j}}.
\end{align}

This may be extended change to the logit scale for numerical stability
and includes regression coefficients.

\begin{align}
\textrm{logit}(\psi) &= \mu_{\psi} \\
\textrm{logit}(p)  &=  \mu_{p} 
\end{align}

\begin{itemize}
\tightlist
\item
  Sampling unit \(j\), which can be time or repeatedly sampled through
  time or space for \(j \in 1, 2, \ldots N_\textrm{units}\).
\item
  indexing vector, \(jj\) that is used with loop over vectors,
  specifically \(jj_\psi\) and \(jj_p\).
\item
  Matrix of coefficients for sampling units \(\beta_\psi\) and
  \(\delta_p\)
\item
  Also, include hyper-parameters based upon syntax of Stan Development
  Team (2022) \S 1.13
\end{itemize}

\begin{align}
\mu_{\psi_{jj[n]}} & \sim X \beta_{jj[n]} \\
\mu_{p_{jj[n]}} & \sim V \delta_{jj[n]}
\end{align}

The coefficients then have their own hierarchy of modeling:
\begin{align}
\beta_{j}  & \sim \textrm{multivariate normal}(\beta^\star, \Sigma_{\psi}) \\
\delta_{j} & \sim \textrm{multivariate normal}(\delta^\star, \Sigma_{p}) \\
\end{align}

The covariance matrices, \(\Sigma_\psi\) and \(\Sigma_p\) are defined in
terms of coefficient scales \(\tau_\psi\) and \(\tau_p\) and correlation
matrices \(\Omega_\psi\) and \(\Omega_p\). The coefficient scales are
defined as \(\tau_\psi = \sqrt{\Sigma_{\psi_{k,k}}}\) and
\(\tau_p = \sqrt{\Sigma_{p_{k,k}}}\). The correlation matrices are
defined as \begin{align}
\Omega_{\psi_{i,j}} &= \frac{\Sigma_{\psi_{i,j}}}{\tau_{\psi_i} \tau_{\psi_j}}~\textrm{and} \\ \Omega_{p_{i,j}} &= \frac{\Sigma_{p_{i,j}}}{\tau_{p_i} \tau_{p_j}}.
\end{align}

Both \(\beta^\star\) and \(\delta^2\) are given weekly-informative
priors: \begin{align}
\beta^\star &\sim \textrm{normal}(0, 2) ~\textrm{and} \\
\delta^\star &\sim \textrm{normal}(0, 2).
\end{align}

Likewise, the \(\tau\) parameters are given a weakly informative prior
from the half-Cauchy distribution:

\begin{align}
\tau_\psi &\sim \textrm{Cauchy}(0, 2.5) ~ \textrm{contrained by} ~ \tau_\psi > 0 ~ \textrm{and} \\
\tau_p &\sim \textrm{Cauchy}(0, 2.5) ~ \textrm{contrained by} ~ \tau_p > 0.
\end{align}

We used Lewandowski, Kurowick, and Joe (LKJ) priors for the for the
correlation matrices as defined by Lewandowski \emph{et al.} (2009) with
\(\nu_psi > 1\) and \(\nu_p > 1\)

\begin{align}
\Sigma_\psi \sim \textrm{LKJCrr}(\nu_\psi) ~ \textrm{and} \\
\Sigma_p \sim \textrm{LKJCrr}(\nu_p)
\end{align}

Notes about indexing

\begin{itemize}
\tightlist
\item
  The two-levels become confusing, often the same by coincidence.
\item
  Especially for \(\psi\)-level parameters and \(p^\star\)-level
  parameters
\item
  Important to think about groupings, often have problems when we were
  coding
\end{itemize}

\hypertarget{formal-3-level-model}{%
\subsection{Formal 3-level model}\label{formal-3-level-model}}

\begin{itemize}
\item
  Includes sub-sampling Mordecai \emph{et al.} (2011)

  \begin{itemize}
  \tightlist
  \item
    Based upon Dorazio \& Erickson (2018) for syntax
  \item
    Generically, (1) is site occupied? (2) Is species present for site?
    (3) Did sub-sampling detect the species?
  \item
    Within context of eDNA, Three levels (see figure reprinted from
    Erickson \emph{et al.} (2019)).
  \item
    For eDNA (1) Is eDNA at the site? (2) Is eDNA in sample and
    successfully exacted? and (3) Did molecular tool such as qPCR detect
    eDNA?
  \item
    We insert \(\theta\) for middle-level.
  \end{itemize}
\end{itemize}

Like previous section, we base our formal nation upon Dorazio \&
Erickson (2018) for the occupancy model and Stan Development Team (2022)
\S 1.13 for the hierarchical model. We also include our Stan variable
names in \texttt{code\ format} because tracking indexing with Stan was a
large challenge we faced when implementing this model. The model has
units \(i\) that may a region of interest in spatial, temporal, or both
where \(i \in 1, 2, \ldots N_\textrm{units}\) (in Stan code, this this
\texttt{n\_units}). For example, multiple lakes could be visited, the
same lake could be visited multiple times, or multiple lakes could be
visited multiple times. The \(i^\textrm{th}\) unit may be occupied
(\(Z_i = 1\)) or not occupied (\(Z_i = 0\)) with probability \(\psi_i\):

\begin{align}
Z_i &\sim \textrm{Bernoulli}(\psi_i).
\end{align}

The \(i^\textrm{th}\) unit has \(j_i\) samples taken from the site,
\(j_i \in 1, 2, \ldots N_\textrm{revisits: i}\)
(\texttt{n\_samples{[}unit\ index{]}}). A lowercase \(z_i\) denotes the
realized occupancy (\(z_i\) = 1) or non-occupancy (\(z_i\) = 0) at a
unit (\texttt{any\_seen{[}unit\ index{]}}).

The latent, sample occurrence \(a_{i,j}\) is conditional upon \(z_i\)
and (denoted with the vertical bar symbol, \(|\)). Lastly,
\(\theta_{i,j}\) is the sample probability for the \(j^\textrm{th}\)
sample at the \(i^\textrm{th}\) unit.

\begin{align}
A_{i,j} | z_i &\sim \textrm{Bernoulli}(z_i \theta)
\end{align}

A lowercase \(a_{i,j}\) denotes the realized occupancy (\(a_{i,j}\) = 1)
or non-occupancy (\(a_{i,j}\) = 0) at a sample
(\texttt{sample\_seen{[}unit\ index{]}}) Within each sample, there are
\(k_{i,j}\) sub-samples within the unit,
\(k_{i,j} \in 1, 2, \ldots N_\textrm{subsamples: i, j}\)
(\texttt{k\_samples}). Each unit may have its own revisits and each
revisit to a unit may have its own number of subsamples. Hence, there
are can be subscripted subscripts.

\(k\) may be summed for each unit and then written as a vector, \(K\)
for all units (\texttt{k\_samples} that is the same length as the total
number of observations in the data frame, \(N_\textrm{total}\) or
\texttt{total\_observations}). A sampling event may have a detection
(\(Y\_{i,j,k} = 1\)) or a non-detection (\(Y_{i,j,k} = 0\)).

The observation \(Y_{i,j,k}\) is conditional upon \(a_{i,j}\) (denoted
with the vertical bar symbol, \(|\)). Lastly, \(p_{i,j,k}\) is the
detection probability for the \(k^\textrm{th}\) sub-sample in the
\(j^\textrm{th}\) sample at the \(i^\textrm{th}\) unit.

\begin{align}
Y_{i,j,k} | a_{i,j} &\sim \textrm{Binomail}(k_{j,k}, a_{i,j} p_{i,j,k})
\end{align}

\begin{itemize}
\tightlist
\item
  This model may be expanded to include regression coefficients.
\item
  We use logit scale, other common choice is probit scale (e.g., Dorazio
  \& Erickson 2018).
\item
  Both similar, logit has slightly wider tails to the distribution
  (Finney 1952), and works more efficiently with Stan due to build in
  and optimized functions.
\end{itemize}

\begin{align}
\textrm{logit}(\psi) &= \mu_{\psi} \\
\textrm{logit}(\theta) &= \mu_{\theta} \\
\textrm{logit}(p)  &=  \mu_{p}
\end{align}

\begin{itemize}
\tightlist
\item
  Can regressors with \(\mu\)s
\item
  Predictor matrices \(X\), \(W\), and \(V\) as well a regression
  coefficients \(\beta\), \(\alpha\), and \(\delta\), that can include
  an error term:
\end{itemize}

\begin{align}
\mu_{\psi} & \sim X \beta \\
\mu_{\theta} & \sim W \alpha \\
\mu_{p} & \sim V \delta
\end{align}

\begin{itemize}
\tightlist
\item
  We can assume \(\beta\), \(\alpha\), and \(\delta\) come from a
  multivariate normal distribution:
\item
  Hierarchical model, adapting notation of Stan Development Team (2022)
  to use use a star symbol (\(^\star\)) for the second level hierarchy
  avoid confusion with too many parameters (i.e., the biology hierarchy
  crossed with the statistical hierarchy).
\item
  \(\Sigma\) are correlated error terms
\end{itemize}

\begin{align}
\beta  & \sim \textrm{multivariate normal}(X^\star \beta^\star, \Sigma_{\psi}) \\
\delta & \sim \textrm{multivariate normal}(V^\star \delta^\star, \Sigma_{p})
\end{align}

\begin{itemize}
\tightlist
\item
  We present two models for both computational efficiency and
  statistical identifiability, specifically one model with
  \(\Sigma_{p}\) and one model without \(\Sigma_{p}\).
\item
  The and correlation matrices \(\Omega_\psi\) and \(\Omega_p\) are
  proportional to the covariance matrices, \(\Sigma_\psi\) and
  \(\Sigma_p\):
\end{itemize}

\begin{align}
\Omega_{\psi_{i,j}} &\propto \Sigma_{\psi_{i,j}}~\textrm{and} \\
\Omega_{p_{i,j}} &\propto \Sigma_{p_{i,j}}.
\end{align}

This may be extended change to the logit scale for numerical stability
and includes regression coefficients.

\begin{align}
\textrm{logit}(\psi) &= \mu_{\psi} \\
\textrm{logit}(\alpha) &= \mu_{\alpha} \\
\textrm{logit}(p)  &=  \mu_{p} 
\end{align}

\begin{itemize}
\tightlist
\item
  Sampling unit \(j\), which can be time or repeatedly sampled through
  time or space for \(j \in 1, 2, \ldots N_\textrm{units}\).
\item
  indexing vector, \(jj\) that is used with loop over vectors,
  specifically \(jj_\psi\), \(jj_\alpha\), and \(jj_p\).
\item
  Matrix of coefficients for sampling units \(\beta_\psi\) and
  \(\delta_p\)
\item
  Also, include hyper-parameters based upon syntax of Stan Development
  Team (2022) \S 1.13
\end{itemize}

\begin{align}
\mu_{\psi_{jj[n]}} & \sim X \beta_{jj[n]} \\
\mu_{\alpha_{jj[n]}} & \sim X \alpha{jj[n]} \\
\mu_{p_{jj[n]}} & \sim V \delta_{jj[n]}
\end{align}

The coefficients then have their own hierarchy of modeling:
\begin{align}
\beta_{j}  & \sim \textrm{multivariate normal}(\beta^\star, \Sigma_{\psi}) \\
\alpha{j}  & \sim \textrm{multivariate normal}(\alpha^\star, \Sigma_{\alpha}) \\
\delta_{j} & \sim \textrm{multivariate normal}(\delta^\star, \Sigma_{p})
\end{align}

The covariance matrices, \(\Sigma_\psi\), \(\Sigma_\theta\), and
\(\Sigma_p\) are defined in terms of coefficient scales \(\tau_\psi\),
\(\tau_\theta\), and \(\tau_p\) and correlation matrices
\(\Omega_\psi\), \(\Omega_\theta\), and \(\Omega_p\). The coefficient
scales are defined as \(\tau_\psi = \sqrt{\Sigma_{\psi_{k,k}}}\),
\(\tau_\theta = \sqrt{\Sigma_{\theta{k,k}}}\), and
\(\tau_p = \sqrt{\Sigma_{p_{k,k}}}\). The correlation matrices are
defined as \begin{align}
\Omega_{\psi_{i,j}} &= \frac{\Sigma_{\psi_{i,j}}}{\tau_{\psi_i} \tau_{\psi_j}}, \\
\Omega_{\theta{i,j}} &= \frac{\Sigma_{\theta_{i,j}}}{\tau_{\theta_i} \tau_{\theta_i}},~\textrm{and} \\\Omega_{p_{i,j}} &= \frac{\Sigma_{p_{i,j}}}{\tau_{p_i} \tau_{p_j}}.
\end{align}

Both \(\beta^\star\), \(\alpha^\star\) and \(\delta^star\) are given
weekly-informative priors: \begin{align}
\beta^\star &\sim \textrm{normal}(0, 2), \\
\alpha^\star &\sim \textrm{normal}(0, 2), ~\textrm{and} \\
\delta^\star &\sim \textrm{normal}(0, 2).
\end{align}

Likewise, the \(\tau\) parameters are given a weakly informative prior
from the half-Cauchy distribution:

\begin{align}
\tau_\psi &\sim \textrm{Cauchy}(0, 2.5) ~ \textrm{contrained by} ~ \tau_\psi > 0, \\
\tau_\theta &\sim \textrm{Cauchy}(0, 2.5) ~ \textrm{contrained by} ~ \theta\psi > 0,~\textrm{and} \\
\tau_p &\sim \textrm{Cauchy}(0, 2.5) ~ \textrm{contrained by} ~ \tau_p > 0.
\end{align}

We used Lewandowski, Kurowick, and Joe (LKJ) priors for the for the
correlation matrices as defined by Lewandowski \emph{et al.} (2009) with
\(\nu_psi > 1\), \(\nu_theta > 1\), and \(\nu_p > 1\)

\begin{align}
\Sigma_\psi \sim \textrm{LKJCrr}(\nu_\psi), \\
\Sigma_\theta \sim \textrm{LKJCrr}(\nu_\theta), ~ \textrm{and} \\
\Sigma_p \sim \textrm{LKJCrr}(\nu_p)
\end{align}

Notes about indexing

\begin{itemize}
\tightlist
\item
  The three-levels become confusing, second statistical-levels can often
  the same by coincidence.
\item
  Especially for \(\psi\)-level parameters and \(\theta^\star\)
  \(p^\star\)-level parameters
\item
  Important to think about groupings, often have problems when we were
  coding
\end{itemize}

\hypertarget{numerical-implementation}{%
\subsection{Numerical implementation}\label{numerical-implementation}}

\begin{itemize}
\tightlist
\item
  Required development version of RStan as of writing, thus we used
  \texttt{rcmdstan}, specifically because of the \texttt{reduce\_sum()}
  function did not appear in Stan until version 2.23 and RStan version
  2.21 was the stable version of RStan. Likewise, \texttt{rcmdstan}
  allowed us to use bothin within and amoung chain parallelizaiton.
\item
  Onion method (Lewandowski \emph{et al.} 2009) because default Stan
  option LKJ prior method does not scale to large matrices
\item
  Within chain parallel requiring more recent versions of Stan than
  currently \texttt{RStan} version 2.21 (Stan Development Team 2021)
\item
  use the \texttt{reduce\_sum()} function, which allows parallel chains
  to calculate log probability distributions
\item
  used Stan version 2.29 (Stan Development Team 2022), which we called
  through Gabry \& Cesnovar (2022)
\item
  Tested using 40 cores on a local server, for test cases 10 cores per
  thread worked best for our models
\end{itemize}

\hypertarget{example-applications-of-model}{%
\section{Example applications of
model}\label{example-applications-of-model}}

\hypertarget{level-simulated-data}{%
\subsection{2-level simulated data}\label{level-simulated-data}}

\begin{Shaded}
\begin{Highlighting}[]
\NormalTok{knitr}\SpecialCharTok{::}\FunctionTok{include\_graphics}\NormalTok{(}\StringTok{"sim\_example\_figures/known\_spp\_matrix.jpg"}\NormalTok{)}
\NormalTok{knitr}\SpecialCharTok{::}\FunctionTok{include\_graphics}\NormalTok{(}\StringTok{"sim\_example\_figures/omega\_psi\_plot.jpg"}\NormalTok{)}
\end{Highlighting}
\end{Shaded}

\begin{figure}

{\centering \subfloat[Simulated (known) correlations\label{fig:unnamed-chunk-2-1}]{\includegraphics[width=0.5\linewidth]{sim_example_figures/known_spp_matrix} }\subfloat[Estimated correlations\label{fig:unnamed-chunk-2-2}]{\includegraphics[width=0.5\linewidth]{sim_example_figures/omega_psi_plot} }

}

\caption{Species-level occpuancy correlations .}\label{fig:unnamed-chunk-2}
\end{figure}

\begin{Shaded}
\begin{Highlighting}[]
\NormalTok{knitr}\SpecialCharTok{::}\FunctionTok{include\_graphics}\NormalTok{(}\StringTok{"sim\_example\_figures/known\_detection\_matrix.jpg"}\NormalTok{)}
\NormalTok{knitr}\SpecialCharTok{::}\FunctionTok{include\_graphics}\NormalTok{(}\StringTok{"sim\_example\_figures/omega\_p\_plot.jpg"}\NormalTok{)}
\end{Highlighting}
\end{Shaded}

\begin{figure}

{\centering \subfloat[Simulated (known) detection correlations\label{fig:unnamed-chunk-3-1}]{\includegraphics[width=0.5\linewidth]{sim_example_figures/known_detection_matrix} }\subfloat[Estimated correlations\label{fig:unnamed-chunk-3-2}]{\includegraphics[width=0.5\linewidth]{sim_example_figures/omega_p_plot} }

}

\caption{Detection probability correlations .}\label{fig:unnamed-chunk-3}
\end{figure}

\hypertarget{level-bird-example}{%
\subsection{2-level bird example}\label{level-bird-example}}

\hypertarget{level-parasite-model}{%
\subsection{2-level parasite model}\label{level-parasite-model}}

\hypertarget{level-simulated-model}{%
\subsection{3-level simulated model}\label{level-simulated-model}}

\hypertarget{level-repeated-visits}{%
\subsection{3-level repeated visits}\label{level-repeated-visits}}

\hypertarget{level-edna-example}{%
\subsection{3-level eDNA example}\label{level-edna-example}}

\hypertarget{discussion}{%
\section{Discussion}\label{discussion}}

\begin{enumerate}
\def\labelenumi{\arabic{enumi}.}
\tightlist
\item
  Current application
\item
  Next steps
\item
  Future research questions
\item
  Implications for broader ecological literature.
\end{enumerate}

\hypertarget{acknowledgments}{%
\section{Acknowledgments}\label{acknowledgments}}

We thank the USGS Biological Threats and Invasive Species Program for
funding as well as the Great Lakes Restoration Initiative. Any use of
trade, firm, or product names is for descriptive purposes only and does
not imply endorsement by the U.S. Government.

\hypertarget{references}{%
\section*{References}\label{references}}
\addcontentsline{toc}{section}{References}

\hypertarget{refs}{}
\begin{CSLReferences}{1}{0}
\leavevmode\vadjust pre{\hypertarget{ref-Dorazio_2017}{}}%
Dorazio, R.M. \& Erickson, R.A. (2018). {ednaoccupancy: An R package for
multiscale occupancy modelling of environmental DNA data}.
\emph{Molecular Ecology Resources}, \textbf{18}, 368--380. Retrieved
from \url{https://doi.org/10.1111/1755-0998.12735}

\leavevmode\vadjust pre{\hypertarget{ref-dorazio2005estimating}{}}%
Dorazio, R.M. \& Royle, J.A. (2005a). Estimating size and composition of
biological communities by modeling the occurrence of species.
\emph{Journal of the American Statistical Association}, \textbf{100},
389--398.

\leavevmode\vadjust pre{\hypertarget{ref-Dorazio_2005}{}}%
Dorazio, R.M. \& Royle, J.A. (2005b). Estimating size and composition of
biological communities by modeling the occurrence of species.
\emph{Journal of the American Statistical Association}, \textbf{100},
389--398. Retrieved from
\url{https://doi.org/10.1198/016214505000000015}

\leavevmode\vadjust pre{\hypertarget{ref-erickson2019sampling}{}}%
Erickson, R.A., Merkes, C.M. \& Mize, E.L. (2019). Sampling designs for
landscape-level eDNA monitoring programs. \emph{Integrated Environmental
Assessment and Management}, \textbf{15}, 760--771.

\leavevmode\vadjust pre{\hypertarget{ref-erickson2022fishstan}{}}%
Erickson, R.A., Stich, D.S. \& Hebert, J.L. (2022). fishStan:
Hierarchical bayesian models for fisheries. \emph{Journal of Open Source
Software}, \textbf{7}, 3444.

\leavevmode\vadjust pre{\hypertarget{ref-fishStan}{}}%
Erickson, R.A., Stich, D.S. \& Hebert, J.L. (2020). \emph{fishStan:
Hierarchical bayesian models for fisheries version 2.0}. U.S. Geological
Survey software release, Reston, VA, https://doi.org/10.5066/P9TT3ILO.

\leavevmode\vadjust pre{\hypertarget{ref-finney1952probit}{}}%
Finney, D.J. (1952). \emph{Probit analysis: A statistical treatment of
the sigmoid response curve}. Cambridge university press, Cambridge.

\leavevmode\vadjust pre{\hypertarget{ref-cmdstanr}{}}%
Gabry, J. \& Cesnovar, R. (2022). \emph{Cmdstanr: R interface to
'CmdStan'}.

\leavevmode\vadjust pre{\hypertarget{ref-gelman2006data}{}}%
Gelman, A. \& Hill, J. (2006). \emph{Data analysis using regression and
multilevel/hierarchical models}. Cambridge university press.

\leavevmode\vadjust pre{\hypertarget{ref-latimer2009hierarchical}{}}%
Latimer, A., Banerjee, S., Sang Jr, H., Mosher, E. \& Silander Jr, J.
(2009). Hierarchical models facilitate spatial analysis of large data
sets: A case study on invasive plant species in the northeastern united
states. \emph{Ecology letters}, \textbf{12}, 144--154.

\leavevmode\vadjust pre{\hypertarget{ref-Lewandowski_2009}{}}%
Lewandowski, D., Kurowicka, D. \& Joe, H. (2009). Generating random
correlation matrices based on vines and extended onion method.
\emph{Journal of Multivariate Analysis}, \textbf{100}, 1989--2001.
Retrieved from \url{https://doi.org/10.1016/j.jmva.2009.04.008}

\leavevmode\vadjust pre{\hypertarget{ref-mackenzie2017occupancy}{}}%
MacKenzie, D.I., Nichols, J.D., Royle, J.A., Pollock, K.H., Bailey, L.L.
\& Hines, J.E. (2017). \emph{Occupancy estimation and modeling:
Inferring patterns and dynamics of species occurrence}. Elsevier.

\leavevmode\vadjust pre{\hypertarget{ref-Mordecai_2011}{}}%
Mordecai, R.S., Mattsson, B.J., Tzilkowski, C.J. \& Cooper, R.J. (2011).
Addressing challenges when studying mobile or episodic species:
Hierarchical bayes estimation of occupancy and use. \emph{Journal of
Applied Ecology}, \textbf{48}, 56--66. Retrieved from
\url{https://doi.org/10.1111/j.1365-2664.2010.01921.x}

\leavevmode\vadjust pre{\hypertarget{ref-royle2008hierarchical}{}}%
Royle, J.A. \& Dorazio, R.M. (2008). \emph{Hierarchical modeling and
inference in ecology: The analysis of data from populations,
metapopulations and communities}. Elsevier.

\leavevmode\vadjust pre{\hypertarget{ref-rstan221}{}}%
Stan Development Team. (2021). {RStan}: The {R} interface to {Stan}.
Retrieved from \url{https://mc-stan.org/}

\leavevmode\vadjust pre{\hypertarget{ref-Stan_manual_2022}{}}%
Stan Development Team. (2022). \emph{{Stan Modeling Language Users Guide
and Reference Manual, version 2.29}}. Retrieved from
\url{https://mc-stan.org}

\leavevmode\vadjust pre{\hypertarget{ref-Tobler_2019}{}}%
Tobler, M.W., K'ery, M., Hui, F.K.C., Guillera-Arroita, G., Knaus, P. \&
Sattler, T. (2019). Joint species distribution models with species
correlations and imperfect detection. \emph{Ecology}, \textbf{100}.
Retrieved from \url{https://doi.org/10.1002/ecy.2754}

\end{CSLReferences}

\end{document}
